Siendo $n$ el número total de celdas de un mapa y $d$ el número de defensas que tenemos que colocar:

En un algoritmo voraz las operación crítica es la operación de selección.

Si no preordenamos las defensas, la función de selección será de orden $O(n)$, ya que la función debe recorrer toda la lista de casillas para determinar cual es la mejor.

Esa función será ejecutada un total de $d$ veces en el mejor caso, ya que todas las defensas se podrían colocar en la casilla que se seleccionase.
En el peor caso se ejecutará un total de n veces, si se acaban seleccionando todas las casillas.

En cada iteración se elimina una casilla de la lista por lo que cada vez el tiempo de ejecución de la función de selección disminuye en una comparación.
Por lo tanto:

\[ t(n) = t(n-1) + 1 = \displaystyle\sum_{i=1}^{n}{n-i} = \frac{n^2 -n}{2} \in O(n^2) \]

En el caso de que se ordenen la lista de casilla previamente obtenemos un tiempo de selección de $O(1)$, que al repetirse como máximo $n$ veces tendríamos un tiempo de $O(n)$ para
todos los algoritmos en los que se preordene la lista de casillas.
Por lo tanto, para decantarnos por uno de ellos tendremos que analizar el orden de complejidad de los propios algoritmos de ordenación.

Suponiendo un mapa cuadrado, el algoritmo de ordenación por fusión siempre podrá dividir el vector de casillas de forma equitativa, por lo que, tal 
y como se analizó en clase obtenemos que el orden del algoritmo en el peor caso es de:

Si $n>n_0$, donde en este caso $n_0 = 3$
\[t(n) = 2t(\frac{n}{2})+n\] 
\[t(n) \in O(n\log{n})\]

Analizando el algoritmo de ordenacion rápida en el peor caso, (el pivote queda en alguno de los extremos) obtenemos un orden de $O(n^2)$
pero en el caso promedio y en el mejor caso obtenemos un orden perteneciente a $O(n\log{n})$

En caso de usar un montículo la función de selección no es $O(1)$ si no de $O(\log{n})$.
La inserción de elementos en un montículo es de orden $O(log{n})$ en el peor caso, como hay que insertar $n$ elementos acaba siendo $O(n\log{n})$.
Y como en el peor caso se extraen $n$ elementos del montículo, pero como el arbol se va haciendo más pequeño en cada iteración el orden es $O(\log{n!})$

Como en el caso peor en la inserción de un montículo tomaremos ese como nuestro algoritmo elegido.